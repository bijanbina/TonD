%PineGreen
\definecolor{ao(english)}{rgb}{0.0, 0.5, 0.0}
\definecolor{airforceblue}{rgb}{0.27, 0.47, 0.6}
\definecolor{Headblue}{rgb}{0.25, 0.37, 0.5}
\definecolor{backBlue}{HTML}{0084A9}
\definecolor{backCirc}{HTML}{F7F7F7}
\definecolor{backGray}{HTML}{545454}
\definecolor{firstLine}{HTML}{1a54ab}
\definecolor{secondLine}{HTML}{ab1919}
\definecolor{thirdLine}{HTML}{157336}


\newcommand{\hwMaincolor}{black}
\newcommand{\hwRulecolor}{black}
\def\chpcolor{airforceblue}
\def\chpcolortxt{RoyalBlue}
\def\sectionfont{\sffamily\LARGE}

\setcounter{secnumdepth}{10}
\newlength{\hwProblemIndent}
\newlength{\hwProblemWidth}
\newlength{\hwPartIndent}
\newlength{\hwPartWidth}
\setlength{\parskip}{0.1cm}

% Margins
\topmargin=-0.45in
\evensidemargin=0in
\oddsidemargin=-0.375in
\textwidth=7.25in
\textheight=9.3in
\headsep=5pt

%Setup section title and margin
\setlength{\hwProblemIndent}{20pt}
\setlength{\hwProblemWidth}{\textwidth}
\setlength{\hwPartIndent}{20pt}
\setlength{\hwPartWidth}{\textwidth}
\addtolength{\hwProblemWidth}{-\hwProblemIndent}
\addtolength{\hwPartWidth}{-5em}


%Colorful Section
\newcommand{\hsectionstrut}{\rule{0pt}{12.5pt}}
\newcommand{\makesectionhead}[2]{%
  {\par\vspace{20pt}%
   \parindent 0pt\raggedleft\sectionfont
   \colorbox{\chpcolor}{%
     \parbox[t]{50pt}{\color{white}\hsectionstrut\hfill#2}%
   }%
   \begin{minipage}[t]{\dimexpr\textwidth-50pt-2\fboxsep\relax}
   \color{\chpcolortxt}\hsectionstrut\hspace{5pt}#1
   \end{minipage} \\
   \vspace{10pt}%
  }
}


\linespread{1.1} % Line spacing
%----------------------------------------------------------------------------------------
%	Header
%----------------------------------------------------------------------------------------
\pagestyle{fancy}
\lhead{\color{\hwMaincolor}\thinfont Summary} % Top left header
\chead{\color{\hwMaincolor}\thinfont Electronic II} % Top center head
\rhead{\color{\hwMaincolor}\thinfont \Places}% Top right header
\lfoot{\color{\hwMaincolor}\lastxmark} % Bottom left footer
\cfoot{} % Bottom center footer
\rfoot{\color{\hwMaincolor}Page\ \thepage\ of\ \protect\pageref{LastPage}} % Bottom right footer
\renewcommand\headrulewidth{0.4pt} % Size of the header rule
\renewcommand\footrulewidth{0.4pt} % Size of the footer rule
\makeatletter
\renewcommand\headrule{\color{\hwRulecolor} \if@fancyplain \let \headrulewidth \plainheadrulewidth \fi \hrule \@height \headrulewidth \@width \headwidth \vskip \headrulewidth }
\renewcommand\footrule{\color{\hwRulecolor} \if@fancyplain \let \footrulewidth \plainfootrulewidth \fi \vskip -\footruleskip \vskip \footrulewidth \hrule \@width \headwidth \@height \footrulewidth \vskip \footruleskip }

\makeatother

\setlength\parindent{0pt} % Removes all indentation from paragraphs
%----------------------------------------------------------------------------------------
%	DOCUMENT STRUCTURE COMMANDS
%	Skip this unless you know what you're doing
%----------------------------------------------------------------------------------------
%\pagenumbering{gobble}
% Header and footer for when a page split occurs within a problem environment
\newcommand{\enterProblemHeader}[1]{
\nobreak\extramarks{#1}{#1 continued on next page\ldots}\nobreak
\nobreak\extramarks{#1 (continued)}{#1 continued on next page\ldots}\nobreak
}

% Header and footer for when a page split occurs between problem environments
\newcommand{\exitProblemHeader}[1]{
\nobreak\extramarks{#1 (continued)}{#1 continued on next page\ldots}\nobreak
\nobreak\extramarks{#1}{}\nobreak
}


\newcommand{\homeworkProblemName}{}
\newenvironment{homeworkProblem}[2]{{
\renewcommand{\homeworkProblemName}{#1 Part} % Assign \homeworkProblemName the name of the problem
\makesectionhead{#2}{#1}% Make a section in the document with the custom problem count
\enterProblemHeader{\homeworkProblemName} % Header and footer within the environment
}}{
\exitProblemHeader{\homeworkProblemName} % Header and footer after the environment
}

\newcommand{\homeworkSectionName}{}
\newenvironment{homeworkSection}[1]{ % New environment for sections within homework problems, takes 1 argument - the name of the section
\renewcommand{\homeworkSectionName}{#1} % Assign \homeworkSectionName to the name of the section from the environment argument
\subsection{\homeworkSectionName} % Make a subsection with the custom name of the subsection
\enterProblemHeader{\homeworkProblemName\ [\homeworkSectionName]} % Header and footer within the environment
}{
\enterProblemHeader{\homeworkProblemName} % Header and footer after the environment
}
      %
     % %
    % % %
   %  %  %
  %       %
 %    %    %
%%%%%%%%%%%%%
\setcounter{section}{1}
%\setcounter{secnumdepth}{3} % Removes default section numbers



%--------------------------------------------------%
%---------------------Dimensioning-----------------%
%--------------------------------------------------%

\tikzset{%
    Cote node/.style={%
        midway,
        sloped,
        fill=backCirc,
        inner sep=1.5pt,
        outer sep=2pt
    },
    Cote arrow/.style={%
        <->,
        >=latex,
        very thin
    }
}

\makeatletter
\NewDocumentCommand{\Cote}{%
    s       % cotation avec les flèches à l'extérieur
    D<>{1.5pt} % offset des traits
    O{.75cm}    % offset de cotation
    m       % premier point
    m       % second point
    m       % étiquette
    D<>{o}  % () coordonnées -> angle
            % h -> horizontal,
            % v -> vertical
            % o or what ever -> oblique
    O{}     % parametre du tikzset
    }{%

    {\tikzset{#8}

    \coordinate (@1) at #4 ;
    \coordinate (@2) at #5 ;

    \if #7v % Cotation verticale
        \coordinate (@0) at ($($#4!.5!#5$) + (#3,0)$) ; 
        \coordinate (@4) at (@0|-@1) ;
        \coordinate (@5) at (@0|-@2) ;
    \else
    \if #7h % Cotation horizontale
        \coordinate (@0) at ($($#4!.5!#5$) + (0,#3)$) ; 
        \coordinate (@4) at (@0-|@1) ;
        \coordinate (@5) at (@0-|@2) ;
    \else % cotation encoche
    \ifnum\pdfstrcmp{\unexpanded\expandafter{\@car#7\@nil}}{(}=\z@
        \coordinate (@5) at ($#7!#3!#5$) ;
        \coordinate (@4) at ($#7!#3!#4$) ;
    \else % cotation oblique    
        \coordinate (@5) at ($#5!#3!90:#4$) ;
        \coordinate (@4) at ($#4!#3!-90:#5$) ;
    \fi\fi\fi

    \draw[very thin,shorten >= #2,shorten <= -2*#2] (@4) -- #4 ;
    \draw[very thin,shorten >= #2,shorten <= -2*#2] (@5) -- #5 ;

    \IfBooleanTF #1 {% avec étoile
    \draw[Cote arrow,-] (@4) -- (@5)
        node[Cote node] {#6\strut};
    \draw[Cote arrow,<-] (@4) -- ($(@4)!-6pt!(@5)$) ;   
    \draw[Cote arrow,<-] (@5) -- ($(@5)!-6pt!(@4)$) ;   
    }{% sans étoile
    \ifnum\pdfstrcmp{\unexpanded\expandafter{\@car#7\@nil}}{(}=\z@
        \draw[Cote arrow] (@5) to[bend right]
            node[Cote node] {#6\strut} (@4) ;
    \else
    \draw[Cote arrow] (@4) -- (@5)
        node[Cote node] {#6\strut};
    \fi
    }}
    }
\makeatother

%-----------------------------------------------------%
%------------------------Show Size--------------------%
%-----------------------------------------------------%
\makeatletter
\newcommand\thefontsize{{ The current font size is: \f@size pt\par}}
\makeatother

%-----------------------------------------------------%
%------------------------Fix Frac---------------------%
%-----------------------------------------------------%


