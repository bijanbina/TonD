$$ A_V = \dfrac{R_C || R_L}{g_m^{-1} + {R_E|| R_{sig}}} ~~ , ~~ V_{CB sat} = -0.5  ~~ , ~~ {\Delta}_1 = V_{CBQ} - V_{CB sat} ~~ , ~~ I_{CQ} - \frac{{\Delta}_2}{R_{ac}} = 0 $$
\setlength{\parindent}{0.5cm} % Default is 15pt.
\lipsum[1-3]
\setlength{\parindent}{0.0cm} 
\begin{minipage}[b]{.5\textwidth}
\lipsum[66]
\begin{tikzpicture}
\tikzset{>=latex}
%lines
\draw [firstLine, thick,name path=line 1] (-1.5,4) node [anchor=west] {~~\footnotesize DC Load Line} -- (3.5,0);
\draw [secondLine, thick,name path=line 2] (-1.5,3) -- (5.5,0) coordinate (D2) ++ (0,0.5) node [anchor=south] {\footnotesize AC Load Line};

%curve
\draw[thick, name path=VBE] (-1.5,0) to [out=85,in=190] ++ (1.5,3);% to [out=5,in=180] (7,3.2);
\draw[thick]  (0,3) -- +(1:5.5cm);

%Coordinate
\draw [<->, thick] (0,5) node [anchor=south] {$I_{C}$} -- (0,0) -- (6.5,0) node [anchor=north west] {$V_{CB}$};
\draw [thick]  (0,0) -- (-2,0);

%Intersections
%\draw [name intersections={of=line 1 and line 2}] [help lines]  let \p1 = (intersection-1) in (intersection-1) -- ++(-\x1,0) coordinate (Icq) node [anchor=east] {$I_{CQ}$};
%\draw [help lines]  let \p1 = (intersection-1) in (intersection-1) -- ++(0,-\y1) coordinate (Vceq) node [anchor=north] {$V_{CEQ}$};
\draw [name intersections={of=line 1 and line 2}] [help lines]  let \p1 = (intersection-1) in (intersection-1) -- ++(-\x1,0) coordinate (Icq);
\draw [help lines]  let \p1 = (intersection-1) in (intersection-1) -- ++(0,-\y1) coordinate (Vceq);
\fill (intersection-1) circle (1.5pt) node [anchor=south west] {Q};

\def \Vsat {-0.5}
\path [name path=A--B] (\Vsat,0) -| (\Vsat,3);
\path [name intersections={of=A--B and VBE}];
\draw [help lines]  (intersection-1) -- (\Vsat,0);
\node [anchor=north east, help lines] at (0.\Vsat,0) {$-0.5$};

%Annotate
\Cote{(\Vsat,0)}{(Vceq)}{${\Delta}_1$}
\Cote{(Vceq)}{(D2)}{${\Delta}_2$}

\end{tikzpicture}\\
\end{minipage}%
\noindent
\setlength{\parindent}{0.0cm} 
\hspace*{-\parindent}%
\begin{minipage}[b]{.5\textwidth}
\centering
~\vspace{-3em}
\subsection{Common Base}\vspace{0.5em}
$$ A_V = \dfrac{R_C || R_L}{g_m^{-1} + {R_E|| R_{sig}}} ~~ , ~~ V_{CB sat} = -0.5  ~~ , ~~ {\Delta}_1 = V_{CBQ} - V_{CB sat} ~~ , ~~ I_{CQ} - \frac{{\Delta}_2}{R_{ac}} = 0 $$
$$ R_{in} = \frac{1}{g_m} + \frac{R_1 || R_2}{\beta + 1} ~~ , ~~ R_{in} = \frac{R_L + r_{o}}{1+g_m r_o} $$
\par
\setlength{\parindent}{0.5cm} % Default is 15pt.
\lipsum[1-3]
\setlength{\parindent}{0.0cm} 
\begin{minipage}[b]{.5\textwidth}
\lipsum[66]
\begin{tikzpicture}
\tikzset{>=latex}
%lines
\draw [firstLine, thick,name path=line 1] (-1.5,4) node [anchor=west] {~~\footnotesize DC Load Line} -- (3.5,0);
\draw [secondLine, thick,name path=line 2] (-1.5,3) -- (5.5,0) coordinate (D2) ++ (0,0.5) node [anchor=south] {\footnotesize AC Load Line};

%curve
\draw[thick, name path=VBE] (-1.5,0) to [out=85,in=190] ++ (1.5,3);% to [out=5,in=180] (7,3.2);
\draw[thick]  (0,3) -- +(1:5.5cm);

%Coordinate
\draw [<->, thick] (0,5) node [anchor=south] {$I_{C}$} -- (0,0) -- (6.5,0) node [anchor=north west] {$V_{CB}$};
\draw [thick]  (0,0) -- (-2,0);

%Intersections
%\draw [name intersections={of=line 1 and line 2}] [help lines]  let \p1 = (intersection-1) in (intersection-1) -- ++(-\x1,0) coordinate (Icq) node [anchor=east] {$I_{CQ}$};
%\draw [help lines]  let \p1 = (intersection-1) in (intersection-1) -- ++(0,-\y1) coordinate (Vceq) node [anchor=north] {$V_{CEQ}$};
\draw [name intersections={of=line 1 and line 2}] [help lines]  let \p1 = (intersection-1) in (intersection-1) -- ++(-\x1,0) coordinate (Icq);
\draw [help lines]  let \p1 = (intersection-1) in (intersection-1) -- ++(0,-\y1) coordinate (Vceq);
\fill (intersection-1) circle (1.5pt) node [anchor=south west] {Q};

\def \Vsat {-0.5}
\path [name path=A--B] (\Vsat,0) -| (\Vsat,3);
\path [name intersections={of=A--B and VBE}];
\draw [help lines]  (intersection-1) -- (\Vsat,0);
\node [anchor=north east, help lines] at (0.\Vsat,0) {$-0.5$};

%Annotate
\Cote{(\Vsat,0)}{(Vceq)}{${\Delta}_1$}
\Cote{(Vceq)}{(D2)}{${\Delta}_2$}

\end{tikzpicture}\\
\end{minipage}%
\noindent
\setlength{\parindent}{0.0cm} 
\hspace*{-\parindent}%
\begin{minipage}[b]{.5\textwidth}
\centering
~\vspace{-3em}
\subsection{Common Base}\vspace{0.5em}
$$ A_V = \dfrac{R_C || R_L}{g_m^{-1} + {R_E|| R_{sig}}} ~~ , ~~ V_{CB sat} = -0.5  ~~ , ~~ {\Delta}_1 = V_{CBQ} - V_{CB sat} ~~ , ~~ I_{CQ} - \frac{{\Delta}_2}{R_{ac}} = 0 $$
$$ R_{in} = \frac{1}{g_m} + \frac{R_1 || R_2}{\beta + 1} ~~ , ~~ R_{in} = \frac{R_L + r_{o}}{1+g_m r_o} $$
\par
\setlength{\parindent}{0.5cm} % Default is 15pt.
\lipsum[1-3]
\setlength{\parindent}{0.0cm} 
\begin{minipage}[b]{.5\textwidth}
\lipsum[66]
\begin{tikzpicture}
\tikzset{>=latex}
%lines
\draw [firstLine, thick,name path=line 1] (-1.5,4) node [anchor=west] {~~\footnotesize DC Load Line} -- (3.5,0);
\draw [secondLine, thick,name path=line 2] (-1.5,3) -- (5.5,0) coordinate (D2) ++ (0,0.5) node [anchor=south] {\footnotesize AC Load Line};

%curve
\draw[thick, name path=VBE] (-1.5,0) to [out=85,in=190] ++ (1.5,3);% to [out=5,in=180] (7,3.2);
\draw[thick]  (0,3) -- +(1:5.5cm);

%Coordinate
\draw [<->, thick] (0,5) node [anchor=south] {$I_{C}$} -- (0,0) -- (6.5,0) node [anchor=north west] {$V_{CB}$};
\draw [thick]  (0,0) -- (-2,0);

%Intersections
%\draw [name intersections={of=line 1 and line 2}] [help lines]  let \p1 = (intersection-1) in (intersection-1) -- ++(-\x1,0) coordinate (Icq) node [anchor=east] {$I_{CQ}$};
%\draw [help lines]  let \p1 = (intersection-1) in (intersection-1) -- ++(0,-\y1) coordinate (Vceq) node [anchor=north] {$V_{CEQ}$};
\draw [name intersections={of=line 1 and line 2}] [help lines]  let \p1 = (intersection-1) in (intersection-1) -- ++(-\x1,0) coordinate (Icq);
\draw [help lines]  let \p1 = (intersection-1) in (intersection-1) -- ++(0,-\y1) coordinate (Vceq);
\fill (intersection-1) circle (1.5pt) node [anchor=south west] {Q};

\def \Vsat {-0.5}
\path [name path=A--B] (\Vsat,0) -| (\Vsat,3);
\path [name intersections={of=A--B and VBE}];
\draw [help lines]  (intersection-1) -- (\Vsat,0);
\node [anchor=north east, help lines] at (0.\Vsat,0) {$-0.5$};

%Annotate
\Cote{(\Vsat,0)}{(Vceq)}{${\Delta}_1$}
\Cote{(Vceq)}{(D2)}{${\Delta}_2$}

\end{tikzpicture}\\
\end{minipage}%
\noindent
\setlength{\parindent}{0.0cm} 
\hspace*{-\parindent}%
\begin{minipage}[b]{.5\textwidth}
\centering
~\vspace{-3em}
\subsection{Common Base}\vspace{0.5em}
$$ A_V = \dfrac{R_C || R_L}{g_m^{-1} + {R_E|| R_{sig}}} ~~ , ~~ V_{CB sat} = -0.5  ~~ , ~~ {\Delta}_1 = V_{CBQ} - V_{CB sat} ~~ , ~~ I_{CQ} - \frac{{\Delta}_2}{R_{ac}} = 0 $$
$$ R_{in} = \frac{1}{g_m} + \frac{R_1 || R_2}{\beta + 1} ~~ , ~~ R_{in} = \frac{R_L + r_{o}}{1+g_m r_o} $$
\par
\setlength{\parindent}{0.5cm} % Default is 15pt.
\lipsum[1-3]
\setlength{\parindent}{0.0cm} 
\begin{minipage}[b]{.5\textwidth}
\lipsum[66]
\begin{tikzpicture}
\tikzset{>=latex}
%lines
\draw [firstLine, thick,name path=line 1] (-1.5,4) node [anchor=west] {~~\footnotesize DC Load Line} -- (3.5,0);
\draw [secondLine, thick,name path=line 2] (-1.5,3) -- (5.5,0) coordinate (D2) ++ (0,0.5) node [anchor=south] {\footnotesize AC Load Line};

%curve
\draw[thick, name path=VBE] (-1.5,0) to [out=85,in=190] ++ (1.5,3);% to [out=5,in=180] (7,3.2);
\draw[thick]  (0,3) -- +(1:5.5cm);

%Coordinate
\draw [<->, thick] (0,5) node [anchor=south] {$I_{C}$} -- (0,0) -- (6.5,0) node [anchor=north west] {$V_{CB}$};
\draw [thick]  (0,0) -- (-2,0);

%Intersections
%\draw [name intersections={of=line 1 and line 2}] [help lines]  let \p1 = (intersection-1) in (intersection-1) -- ++(-\x1,0) coordinate (Icq) node [anchor=east] {$I_{CQ}$};
%\draw [help lines]  let \p1 = (intersection-1) in (intersection-1) -- ++(0,-\y1) coordinate (Vceq) node [anchor=north] {$V_{CEQ}$};
\draw [name intersections={of=line 1 and line 2}] [help lines]  let \p1 = (intersection-1) in (intersection-1) -- ++(-\x1,0) coordinate (Icq);
\draw [help lines]  let \p1 = (intersection-1) in (intersection-1) -- ++(0,-\y1) coordinate (Vceq);
\fill (intersection-1) circle (1.5pt) node [anchor=south west] {Q};

\def \Vsat {-0.5}
\path [name path=A--B] (\Vsat,0) -| (\Vsat,3);
\path [name intersections={of=A--B and VBE}];
\draw [help lines]  (intersection-1) -- (\Vsat,0);
\node [anchor=north east, help lines] at (0.\Vsat,0) {$-0.5$};

%Annotate
\Cote{(\Vsat,0)}{(Vceq)}{${\Delta}_1$}
\Cote{(Vceq)}{(D2)}{${\Delta}_2$}

\end{tikzpicture}\\
\end{minipage}%
\noindent
\setlength{\parindent}{0.0cm} 
\hspace*{-\parindent}%
\begin{minipage}[b]{.5\textwidth}
\centering
\input{Circuits/CB}
\end{minipage}%

\end{minipage}%

\end{minipage}%

\end{minipage}%
