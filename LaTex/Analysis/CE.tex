%\thefontsize 
$$ g_m = \cfrac{\partial I_c}{\partial V_{BE}} = \cfrac{I_c}{V_t} ~~,~~ r_\pi = \cfrac{\beta}{g_m} ~~ , ~~ I_C = I_S ( exp (\dfrac{V_{BE}}{V_T}) + 1)~~ , ~~ h_{oe} = output~admittance ~~ , ~~ h_{fe} = \beta $$
$$AC~Line~Slope:  -g_m ( R_C || R_L )  ~~ , ~~ DC~Load~Line :  V_{CC} - R_C I_{CQ} - V_{CEQ} = 0 $$
$$ A_V = -\dfrac{R_C || R_L}{g_m^{-1} + R_E} $$
\setlength{\parindent}{0.5cm} % Default is 15pt.
\par
Common x is an amplifing configuration which the node x, is isolated from both input source and output node. To compute swing voltage this node and the output node are plotted versus $I_C$. This is done to independ the swing calculation from input alternate. So as said in Common Source configuration this is $I_D$ versus $V_DS$. After writing DC load line from mentioned nodes. We determine AC Line, Slope. So we define AC Line. AC Line is a Line with "AC Line Slope" Slope and pass from Quiscent point. so for common emitter configuration it is $ I_C = - g_m ( R_C || R_L ) V_{CE} $
\lipsum[1-2]
\setlength{\parindent}{0.0cm} 
$ R_o = (r_\pi ~||~ R_E) + ( 1 + g_m (r_\pi ~||~ R_E) )r_o$ or \\
$ R_o = r_o + (r_\pi ~||~ R_E)( 1 + g_m  )$ \\
$ if ( g_m R_o = r_o + (r_\pi ~||~ R_E)( 1 + g_m  )$ \\
\begin{minipage}[t]{.5\textwidth}
\begin{tikzpicture}
\tikzset{>=latex}
%lines
\draw [firstLine, thick,name path=line 1] (0,4) node [anchor=west] {~~\footnotesize DC Load Line} -- (5,0);
\draw [secondLine, thick,name path=line 2] (0,3) -- (7,0) coordinate (D2) ++ (0,0.5) node [anchor=south] {\footnotesize AC Load Line};

%curve
\draw[thick, name path=VBE] (0,0.5) to [out=85,in=185] (1,3);% to [out=5,in=180] (7,3.2);
\draw[thick]  (1,3) -- +(1:6.5cm);

%Coordinate
\draw [<->, rounded corners, thick] (0,5) node [anchor=south] {$I_{C}$} -- (0,0) -- (8,0) node [anchor=north west] {$V_{CE}$};

\draw [name intersections={of=line 1 and line 2}] [help lines]  let \p1 = (intersection-1) in (intersection-1) -- ++(-\x1,0) coordinate (Icq) node [anchor=east] {$I_{CQ}$};
\draw [help lines]  let \p1 = (intersection-1) in (intersection-1) -- ++(0,-\y1) coordinate (Vceq) node [anchor=north] {$V_{CEQ}$};
\fill (intersection-1) circle (1.5pt);

\path [name path=A--B] (0.65,0) -| (0.65,3);
\path [name intersections={of=A--B and VBE}];
\draw [help lines]  (intersection-1) -- (0.65,0);
\node [anchor=north, help lines] at (0.65,0) {$V_{sat}$};


\Cote{(Vceq)}{(0.65,0)}{${\Delta}_1$}
\Cote{(Vceq)}{(D2)}{${\Delta}_2$}

\end{tikzpicture}\\
\end{minipage}%
\noindent
\setlength{\parindent}{0.0cm} 
\hspace*{-\parindent}%
\begin{minipage}[b]{.5\textwidth}
\raggedleft
%\thefontsize 
$$ g_m = \cfrac{\partial I_c}{\partial V_{BE}} = \cfrac{I_c}{V_t} ~~,~~ r_\pi = \cfrac{\beta}{g_m} ~~ , ~~ I_C = I_S ( exp (\dfrac{V_{BE}}{V_T}) + 1)~~ , ~~ h_{oe} = output~admittance ~~ , ~~ h_{fe} = \beta $$
$$AC~Line~Slope:  -g_m ( R_C || R_L )  ~~ , ~~ DC~Load~Line :  V_{CC} - R_C I_{CQ} - V_{CEQ} = 0 $$
$$ A_V = -\dfrac{R_C || R_L}{g_m^{-1} + R_E} $$
\setlength{\parindent}{0.5cm} % Default is 15pt.
\par
Common x is an amplifing configuration which the node x, is isolated from both input source and output node. To compute swing voltage this node and the output node are plotted versus $I_C$. This is done to independ the swing calculation from input alternate. So as said in Common Source configuration this is $I_D$ versus $V_DS$. After writing DC load line from mentioned nodes. We determine AC Line, Slope. So we define AC Line. AC Line is a Line with "AC Line Slope" Slope and pass from Quiscent point. so for common emitter configuration it is $ I_C = - g_m ( R_C || R_L ) V_{CE} $
\lipsum[1-2]
\setlength{\parindent}{0.0cm} 
$ R_o = (r_\pi ~||~ R_E) + ( 1 + g_m (r_\pi ~||~ R_E) )r_o$ or \\
$ R_o = r_o + (r_\pi ~||~ R_E)( 1 + g_m  )$ \\
$ if ( g_m R_o = r_o + (r_\pi ~||~ R_E)( 1 + g_m  )$ \\
\begin{minipage}[t]{.5\textwidth}
\begin{tikzpicture}
\tikzset{>=latex}
%lines
\draw [firstLine, thick,name path=line 1] (0,4) node [anchor=west] {~~\footnotesize DC Load Line} -- (5,0);
\draw [secondLine, thick,name path=line 2] (0,3) -- (7,0) coordinate (D2) ++ (0,0.5) node [anchor=south] {\footnotesize AC Load Line};

%curve
\draw[thick, name path=VBE] (0,0.5) to [out=85,in=185] (1,3);% to [out=5,in=180] (7,3.2);
\draw[thick]  (1,3) -- +(1:6.5cm);

%Coordinate
\draw [<->, rounded corners, thick] (0,5) node [anchor=south] {$I_{C}$} -- (0,0) -- (8,0) node [anchor=north west] {$V_{CE}$};

\draw [name intersections={of=line 1 and line 2}] [help lines]  let \p1 = (intersection-1) in (intersection-1) -- ++(-\x1,0) coordinate (Icq) node [anchor=east] {$I_{CQ}$};
\draw [help lines]  let \p1 = (intersection-1) in (intersection-1) -- ++(0,-\y1) coordinate (Vceq) node [anchor=north] {$V_{CEQ}$};
\fill (intersection-1) circle (1.5pt);

\path [name path=A--B] (0.65,0) -| (0.65,3);
\path [name intersections={of=A--B and VBE}];
\draw [help lines]  (intersection-1) -- (0.65,0);
\node [anchor=north, help lines] at (0.65,0) {$V_{sat}$};


\Cote{(Vceq)}{(0.65,0)}{${\Delta}_1$}
\Cote{(Vceq)}{(D2)}{${\Delta}_2$}

\end{tikzpicture}\\
\end{minipage}%
\noindent
\setlength{\parindent}{0.0cm} 
\hspace*{-\parindent}%
\begin{minipage}[b]{.5\textwidth}
\raggedleft
%\thefontsize 
$$ g_m = \cfrac{\partial I_c}{\partial V_{BE}} = \cfrac{I_c}{V_t} ~~,~~ r_\pi = \cfrac{\beta}{g_m} ~~ , ~~ I_C = I_S ( exp (\dfrac{V_{BE}}{V_T}) + 1)~~ , ~~ h_{oe} = output~admittance ~~ , ~~ h_{fe} = \beta $$
$$AC~Line~Slope:  -g_m ( R_C || R_L )  ~~ , ~~ DC~Load~Line :  V_{CC} - R_C I_{CQ} - V_{CEQ} = 0 $$
$$ A_V = -\dfrac{R_C || R_L}{g_m^{-1} + R_E} $$
\setlength{\parindent}{0.5cm} % Default is 15pt.
\par
Common x is an amplifing configuration which the node x, is isolated from both input source and output node. To compute swing voltage this node and the output node are plotted versus $I_C$. This is done to independ the swing calculation from input alternate. So as said in Common Source configuration this is $I_D$ versus $V_DS$. After writing DC load line from mentioned nodes. We determine AC Line, Slope. So we define AC Line. AC Line is a Line with "AC Line Slope" Slope and pass from Quiscent point. so for common emitter configuration it is $ I_C = - g_m ( R_C || R_L ) V_{CE} $
\lipsum[1-2]
\setlength{\parindent}{0.0cm} 
$ R_o = (r_\pi ~||~ R_E) + ( 1 + g_m (r_\pi ~||~ R_E) )r_o$ or \\
$ R_o = r_o + (r_\pi ~||~ R_E)( 1 + g_m  )$ \\
$ if ( g_m R_o = r_o + (r_\pi ~||~ R_E)( 1 + g_m  )$ \\
\begin{minipage}[t]{.5\textwidth}
\begin{tikzpicture}
\tikzset{>=latex}
%lines
\draw [firstLine, thick,name path=line 1] (0,4) node [anchor=west] {~~\footnotesize DC Load Line} -- (5,0);
\draw [secondLine, thick,name path=line 2] (0,3) -- (7,0) coordinate (D2) ++ (0,0.5) node [anchor=south] {\footnotesize AC Load Line};

%curve
\draw[thick, name path=VBE] (0,0.5) to [out=85,in=185] (1,3);% to [out=5,in=180] (7,3.2);
\draw[thick]  (1,3) -- +(1:6.5cm);

%Coordinate
\draw [<->, rounded corners, thick] (0,5) node [anchor=south] {$I_{C}$} -- (0,0) -- (8,0) node [anchor=north west] {$V_{CE}$};

\draw [name intersections={of=line 1 and line 2}] [help lines]  let \p1 = (intersection-1) in (intersection-1) -- ++(-\x1,0) coordinate (Icq) node [anchor=east] {$I_{CQ}$};
\draw [help lines]  let \p1 = (intersection-1) in (intersection-1) -- ++(0,-\y1) coordinate (Vceq) node [anchor=north] {$V_{CEQ}$};
\fill (intersection-1) circle (1.5pt);

\path [name path=A--B] (0.65,0) -| (0.65,3);
\path [name intersections={of=A--B and VBE}];
\draw [help lines]  (intersection-1) -- (0.65,0);
\node [anchor=north, help lines] at (0.65,0) {$V_{sat}$};


\Cote{(Vceq)}{(0.65,0)}{${\Delta}_1$}
\Cote{(Vceq)}{(D2)}{${\Delta}_2$}

\end{tikzpicture}\\
\end{minipage}%
\noindent
\setlength{\parindent}{0.0cm} 
\hspace*{-\parindent}%
\begin{minipage}[b]{.5\textwidth}
\raggedleft
%\thefontsize 
$$ g_m = \cfrac{\partial I_c}{\partial V_{BE}} = \cfrac{I_c}{V_t} ~~,~~ r_\pi = \cfrac{\beta}{g_m} ~~ , ~~ I_C = I_S ( exp (\dfrac{V_{BE}}{V_T}) + 1)~~ , ~~ h_{oe} = output~admittance ~~ , ~~ h_{fe} = \beta $$
$$AC~Line~Slope:  -g_m ( R_C || R_L )  ~~ , ~~ DC~Load~Line :  V_{CC} - R_C I_{CQ} - V_{CEQ} = 0 $$
$$ A_V = -\dfrac{R_C || R_L}{g_m^{-1} + R_E} $$
\setlength{\parindent}{0.5cm} % Default is 15pt.
\par
Common x is an amplifing configuration which the node x, is isolated from both input source and output node. To compute swing voltage this node and the output node are plotted versus $I_C$. This is done to independ the swing calculation from input alternate. So as said in Common Source configuration this is $I_D$ versus $V_DS$. After writing DC load line from mentioned nodes. We determine AC Line, Slope. So we define AC Line. AC Line is a Line with "AC Line Slope" Slope and pass from Quiscent point. so for common emitter configuration it is $ I_C = - g_m ( R_C || R_L ) V_{CE} $
\lipsum[1-2]
\setlength{\parindent}{0.0cm} 
$ R_o = (r_\pi ~||~ R_E) + ( 1 + g_m (r_\pi ~||~ R_E) )r_o$ or \\
$ R_o = r_o + (r_\pi ~||~ R_E)( 1 + g_m  )$ \\
$ if ( g_m R_o = r_o + (r_\pi ~||~ R_E)( 1 + g_m  )$ \\
\begin{minipage}[t]{.5\textwidth}
\begin{tikzpicture}
\tikzset{>=latex}
%lines
\draw [firstLine, thick,name path=line 1] (0,4) node [anchor=west] {~~\footnotesize DC Load Line} -- (5,0);
\draw [secondLine, thick,name path=line 2] (0,3) -- (7,0) coordinate (D2) ++ (0,0.5) node [anchor=south] {\footnotesize AC Load Line};

%curve
\draw[thick, name path=VBE] (0,0.5) to [out=85,in=185] (1,3);% to [out=5,in=180] (7,3.2);
\draw[thick]  (1,3) -- +(1:6.5cm);

%Coordinate
\draw [<->, rounded corners, thick] (0,5) node [anchor=south] {$I_{C}$} -- (0,0) -- (8,0) node [anchor=north west] {$V_{CE}$};

\draw [name intersections={of=line 1 and line 2}] [help lines]  let \p1 = (intersection-1) in (intersection-1) -- ++(-\x1,0) coordinate (Icq) node [anchor=east] {$I_{CQ}$};
\draw [help lines]  let \p1 = (intersection-1) in (intersection-1) -- ++(0,-\y1) coordinate (Vceq) node [anchor=north] {$V_{CEQ}$};
\fill (intersection-1) circle (1.5pt);

\path [name path=A--B] (0.65,0) -| (0.65,3);
\path [name intersections={of=A--B and VBE}];
\draw [help lines]  (intersection-1) -- (0.65,0);
\node [anchor=north, help lines] at (0.65,0) {$V_{sat}$};


\Cote{(Vceq)}{(0.65,0)}{${\Delta}_1$}
\Cote{(Vceq)}{(D2)}{${\Delta}_2$}

\end{tikzpicture}\\
\end{minipage}%
\noindent
\setlength{\parindent}{0.0cm} 
\hspace*{-\parindent}%
\begin{minipage}[b]{.5\textwidth}
\raggedleft
\input{Circuits/CE}
%\makesectionhead{Mosfet}{Common source}
\end{minipage}%

%\makesectionhead{Mosfet}{Common source}
\end{minipage}%

%\makesectionhead{Mosfet}{Common source}
\end{minipage}%

%\makesectionhead{Mosfet}{Common source}
\end{minipage}%
