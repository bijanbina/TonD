\setlength{\parindent}{0.5cm} % Default is 15pt.
\lipsum[1-3]
\setlength{\parindent}{0.0cm} 
\begin{minipage}[b]{.5\textwidth}
\lipsum[66]
\begin{tikzpicture}
\tikzset{>=latex}
%lines
\draw [firstLine, thick,name path=line 1] (-1.5,4) node [anchor=west] {~~\footnotesize DC Load Line} -- (3.5,0);
\draw [secondLine, thick,name path=line 2] (-1.5,3) -- (5.5,0) coordinate (D2) ++ (0,0.5) node [anchor=south] {\footnotesize AC Load Line};

%curve
\draw[thick, name path=VBE] (-1.5,0) to [out=85,in=190] ++ (1.5,3);% to [out=5,in=180] (7,3.2);
\draw[thick]  (0,3) -- +(1:5.5cm);

%Coordinate
\draw [<->, thick] (0,5) node [anchor=south] {$I_{D}$} -- (0,0) -- (6.5,0) node [anchor=north west] {$V_{DG}$};
\draw [thick]  (0,0) -- (-2,0);

%Intersections
%\draw [name intersections={of=line 1 and line 2}] [help lines]  let \p1 = (intersection-1) in (intersection-1) -- ++(-\x1,0) coordinate (Icq) node [anchor=east] {$I_{CQ}$};
%\draw [help lines]  let \p1 = (intersection-1) in (intersection-1) -- ++(0,-\y1) coordinate (Vceq) node [anchor=north] {$V_{CEQ}$};
\draw [name intersections={of=line 1 and line 2}] [help lines]  let \p1 = (intersection-1) in (intersection-1) -- ++(-\x1,0) coordinate (Icq);
\draw [help lines]  let \p1 = (intersection-1) in (intersection-1) -- ++(0,-\y1) coordinate (Vceq);
\fill (intersection-1) circle (1.5pt) node [anchor=south west] {Q};

\def \Vsat {-0.5}
\path [name path=A--B] (\Vsat,0) -| (\Vsat,3);
\path [name intersections={of=A--B and VBE}];
\draw [help lines]  (intersection-1) -- (\Vsat,0);
\node [anchor=north east, help lines] at (0.\Vsat,0) {$V_{ov}$};

%Annotate
\Cote{(\Vsat,0)}{(Vceq)}{${\Delta}_1$}
\Cote{(Vceq)}{(D2)}{${\Delta}_2$}

\end{tikzpicture}\\
\end{minipage}%
\noindent
\setlength{\parindent}{0.0cm} 
\hspace*{-\parindent}%
\begin{minipage}[b]{.5\textwidth}
\centering

\begin{figure}
\begin{circuitikz}
\tikzstyle{every node}=[font=\small]


%-------------------------------------------------------------------------%
%---------------------------------Parameters------------------------------%
%-------------------------------------------------------------------------%


\def \VccSpace {8}
\def \LevelSpace {2.25}
\pgfmathparse{\VccSpace-1}
\let \VccToSig \pgfmathresult

\ifx\MosOffset\undefined
\def \MosOffset {0}
\fi
\pgfmathparse{\LevelSpace/2+0.25}
\let \SmallElement \pgfmathresult

\pgfmathparse{\VccSpace/2 + 0.5 + \MosOffset}
\let \VccToMos \pgfmathresult %pgfmathparse used to make expression unify
%\def \NormElement {\VccSpace/2 + 0.5}
\def \AnSize {0.5}

\pgfmathparse{\AnSize/2}
\let \AnSpace \pgfmathresult

\pgfmathparse{\AnSize - 0.1pt}
\let \VccLabelSpace \pgfmathresult

\let \VccLabelSpace \pgfmathresult

\coordinate (VccB) at (3,\VccSpace);
\coordinate (Vcc) at  ($ (VccB) + (\LevelSpace,0) $);
\coordinate (GATE_IN) at  ($ (VccB) - (0,\VccToMos) $);
\coordinate (GND_SIG) at  ($ (R_IN)  - (0,2) $);
\coordinate (GND_Load) at  ($ (R_IN)  - (0,2) $);
\coordinate (GND) at  ($ (Vcc) + (0,-\VccSpace) $);
\coordinate (GNDB) at  ($ (VccB) + (0,-\VccSpace) $);

\def \OutputOffset {1.5}

\newcommand{\neTondComment}[2]{\draw [Stealth-] (#1) ++ (+\AnSpace, 0) -- ++(0,+\AnSize) -- ++(+\AnSize,0)  node [anchor=west] {#2}};
\newcommand{\nwTondComment}[2]{\draw [Stealth-] (#1) ++ (-\AnSpace, 0) -- ++(0,+\AnSize) -- ++(-\AnSize,0)  node [anchor=east] {#2}};
\newcommand{\seTondComment}[2]{\draw [Stealth-] (#1) ++ (+\AnSpace, 0) -- ++(0,-\AnSize) -- ++(+\AnSize,0)  node [anchor=west] {#2}};
\newcommand{\swTondComment}[2]{\draw [Stealth-] (#1) ++ (-\AnSpace, 0) -- ++(0,-\AnSize) -- ++(-\AnSize,0)  node [anchor=east] {#2}};

\newcommand{\neTondCommentH}[2]{\draw [Stealth-] (#1) ++ (0, +\AnSpace) -- ++(+\AnSize,0) -- ++(0,+\AnSize)  node [anchor=south] {#2}};
\newcommand{\nwTondCommentH}[2]{\draw [Stealth-] (#1) ++ (0, +\AnSpace) -- ++(-\AnSize,0) -- ++(0,+\AnSize)  node [anchor=south] {#2}};
\newcommand{\seTondCommentH}[2]{\draw [Stealth-] (#1) ++ (0, -\AnSpace) -- ++(+\AnSize,0) -- ++(0,-\AnSize)  node [anchor=north] {#2}};
\newcommand{\swTondCommentH}[2]{\draw [Stealth-] (#1) ++ (0, -\AnSpace) -- ++(-\AnSize,0) -- ++(0,-\AnSize)  node [anchor=north] {#2}};
%\newcommand{\myfrac}[3][0pt]{\genfrac{}{}{}{}{\raisebox{#1}{$#2$}}{\raisebox{-#1}{$#3$}}}

\def \TondVcc{node [anchor=south] {$V_{CC}$} ++(-\VccLabelSpace,0) --  ++(2*\VccLabelSpace,0)}%VCC
\newcommand{\TondOut}[1]{\node [shape=circle,anchor=west,draw,minimum size=0.1cm,inner sep=0pt] at (#1) {}};
\newcommand{\TondOutE}[1]{\node [shape=circle,anchor=east,draw,minimum size=0.1cm,inner sep=0pt] at (#1) {}};
\newcommand{\TondNode}[3]{
\node [shape=circle,minimum size=0.1cm,inner sep=0pt,fill=black] at (#1) {};
\ifthenelse{\equal{⟨#3⟩}{⟨n⟩}}{\node [anchor=south] at (#1) {#2};}{}
\ifthenelse{\equal{⟨#3⟩}{⟨e⟩}}{\node [anchor=west] at (#1) {#2};}{}
\ifthenelse{\equal{⟨#3⟩}{⟨s⟩}}{\node [anchor=north] at (#1) {#2};}{}
\ifthenelse{\equal{⟨#3⟩}{⟨w⟩}}{\node [anchor=east] at (#1) {#2};}{}
\equal{⟨string⟩}{⟨string⟩}
}

%Draw input signal generator on right of node
\newcommand{\TondSigR}[1]{
\coordinate (C_IN) at  ($ (#1) + (\SmallElement,0) $);
\coordinate (R_IN) at  ($ (C_IN) +  (\SmallElement,0) $);
\coordinate (GND_SIG) at  ($ (R_IN)  - (0,2) $);

\draw (GND_SIG) node [ground] {};%GND
\draw (C_IN) to[C=$C$] (#1);%Rsig
\draw (R_IN) to [R=$R_{sig}$] ($(C_IN) -  (0.4,0) $);%Rsig
\draw (GND_SIG) to [sV=$V_{in}$] (R_IN);%Signal
}
%Draw input signal generator on left of node
\newcommand{\TondSigL}[1]{
\coordinate (C_IN) at  ($ (#1) - (\SmallElement,0) $);
\coordinate (R_IN) at  ($ (C_IN) -  (\SmallElement,0) $);
\coordinate (GND_SIG) at  ($ (R_IN)  - (0,2) $);

\draw (GND_SIG) node [ground] {};%GND
\draw (C_IN) to[C=$C$] (#1);%Rsig
\draw (R_IN) to [R=$R_{sig}$] ($(C_IN) +  (0.4,0) $);%Rsig
\draw (GND_SIG) to [sV=$V_{in}$] (R_IN);%Signal
}



%-------------------------------------------------------------------------%
%-----------------------------------Circuit-------------------------------%
%-------------------------------------------------------------------------%


\draw (VccB) \TondVcc;%VCC
\draw (Vcc) \TondVcc;%VCC
\draw (Vcc) ++ (0,-\VccToMos) node[nmos](mos) {};

%-------------Load & Out-------------
\coordinate (Node_out)  at  ($ (mos.drain) + (0,\OutputOffset) $);
\coordinate (V_out)  at  ($ (Node_out)  + (\LevelSpace,0) $);
\coordinate (R_Load)  at  ($ (V_out) - (1,0) $);
\coordinate (GND_Load) at  ($ (R_Load) - (0,2) $);

\draw (R_Load) to [R=$R_{l}$] (GND_Load);%Rl
\node  [ground] at (GND_Load){};%GND
\draw (Node_out) to [C=$C$,*-] (R_Load) to[short] (V_out) node [anchor=north west] {$V_o$};%Rl
\TondOut {V_out};

%---------------Circuit---------------

\node  [ground] at (GND){};%GND
\node  [ground] at (GNDB){};%GND
\draw (mos.source) to[R=$R_S$] (GND) {};%GND
\draw (Vcc) to [R=\small{$R_D$}] (Node_out) -- (mos.drain);%RD
\draw (VccB) to [R=$R_1$] (GATE_IN) node(RG1) {};%RG1
\draw (RG1) to [R=$R_2$] (GNDB);%RG2
\draw (mos.gate) -- (GATE_IN); %Short Circuit
%(mos.source) node[anchor=north] {S};
\nwTondComment{mos.drain}{$r_o$};
\swTondComment{mos.source}{$\frac{1}{g_m}$};

%----------------Signal--------------
\coordinate (C_IN) at  ($ (mos.source) + (\SmallElement,0) $);
\coordinate (R_IN) at  ($ (C_IN) +  (\SmallElement,0) $);
\coordinate (GND_SIG) at  ($ (R_IN)  - (0,2) $);

\draw (GND_SIG) node [ground] {};%GND
\draw (C_IN) to[C=$C$] (mos.source);%Rsig
\draw (R_IN) to [R=$R_{sig}$] ($(C_IN) -  (0.4,0) $);%Rsig
\draw (GND_SIG) to [sV=$V_{in}$,rotation=10] (R_IN);%Signal

\end{circuitikz}
\caption{Mosfet Common Gate}
\end{figure}

\end{minipage}%
