\subsection{Multistage Amplifiers}
\begin{minipage}[b]{.7\textwidth}
\setlength{\abovedisplayshortskip}{15pt}
\setlength{\belowdisplayshortskip}{15pt} 
To calculate multistage gain for a amplifier. first calculate gain from $V_{in}$ to $V_X$ considering next stage load. 
$$ A_1 = \frac{V_X}{V_{in}} = -g_{m1} R_D = -\frac{g_{m1}}{g_{m2}} $$
then the gain from $V_X$ to $V_{out}$ is
$$ A_2 = \frac{V_{out}}{V_{X}} = g_{m2} R_{O} = {g_{m2}}\,({g_{m2} r_{o1} r _{o2}}) $$
so the overall gain is
$$ A = A_1 \times A_2 = \frac{V_{out}}{V_{in}} = {g_{m1}}\,({g_{m2} r_{o1} r _{o2}}) $$
note that for second stage there is no need to consider input resistance as it's already has been calculated as a load of first stage.
\end{minipage}%
\begin{minipage}{.3\textwidth}
\vspace{-25em}
\hspace{3em}
\begin{circuitikz}
\tikzstyle{every node}=[font=\small]


%-------------------------------------------------------------------------%
%---------------------------------Parameters------------------------------%
%-------------------------------------------------------------------------%


\def \BiasSpace {1}
\def \MosSpace {1.5}
\def \MosBase {0}
\def \AnSize {0.5}
\pgfmathparse{\AnSize/2}
\let \AnSpace \pgfmathresult
% \pgfmathparse{\AnSize - 0.1pt}
% \let \VccLabelSpace \pgfmathresult

\coordinate (NMOS1) at (0,\MosBase);
\coordinate (NMOS2) at ($ (NMOS1) + (0,\MosSpace) $);


\input{Circuits/Commands}


%-------------------------------------------------------------------------%
%-----------------------------------Circuit-------------------------------%
%-------------------------------------------------------------------------%


\draw (NMOS1) node[nmos](mos1) {};
\node [anchor=west] at (NMOS1) {M1};
\draw (NMOS2) node[nmos](mos2) {};
\TondNode{mos1.drain}{$V_X$}{e};
\node [anchor=west] at (NMOS2) {M2};
\node  [ground] at (mos1.source){};%GND

%----------------Bias----------------
\coordinate (Bias1) at ($ (mos1.gate) + (-\BiasSpace,0) $);
\coordinate (Bias2) at ($ (mos2.gate) + (-\BiasSpace,0) $);


\draw (mos1.gate)  -- (Bias1);
\node [anchor=east] at ($(Bias1) + (-0.3,0)$) {$V_{in}$};%Rl
\TondOutE {Bias1};

\draw (mos2.gate)  -- (Bias2);
\node [anchor=east] at ($(Bias2) + (-0.3,0)$) {$V_{G1}$};%Rl
\TondOutE {Bias2};


%-----------------Out----------------
\draw (mos2.drain) -- ++(0.5,0) coordinate (OUT1);%Rl
\node [anchor=west] at ($(OUT1) + (0.3,0)$) {$V_{O}$};%Rl
\TondOut {OUT1};

%-------------Annotate---------------
\nwTondComment{mos2.drain}{$R_{out}$};

\end{circuitikz}

\end{minipage}%


\subsection{Current Mirrors}
\subsubsection{Wilson}
$$ 3~Mosfet ~~ , ~~ I_C = \frac{I_{Ref}}{1 + 2 \beta} $$
\par
\subsubsection{Widlar}
$$ 3~Mosfet ~~ , ~~ I_C = \frac{I_{Ref}}{1 + 2 \beta} $$
\par
\subsection{Feedback}
\subsubsection{Current Feedback}
Input Case: To inject or draw currents into or from inputs the feedback network has to be shunt with amplifier.\\
Output Case: To sample currents from output we want to read current ( like amp meter ) so it has to be series with output.\\
\lipsum[7-8]
\subsubsection{Voltage Feedback}
Input Case: To create voltage control amplifier the feedback network has to be series with signal to change input voltage.\\
Output Case: To sample voltage from output we want act like a voltage meter so it has to be shunt with output.
\par
\setlength{\parindent}{0.5cm} % Default is 15pt.
In Common Drain, $R_D$ Is not used as it has very small impact on output gain, but it take down DC bias. so it not used practically
\lipsum[1]
\setlength{\parindent}{0.0cm} 
