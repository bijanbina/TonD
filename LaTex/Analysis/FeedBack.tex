\subsection{Multistage Amplifiers}
To calculate multistage gain for a amplifier. first calculate gain from $V_{in}$ to $V_x$ considering next stage load. then calculate $R_{out}$ of input stage by disconnecting second stage. then put $R_{out}$ series with a voltage source showing amplify voltage. For calcualting next stage gain note that in the nominator of CE, CB and their MOSFET configuration equivalent, the R_C is $_{out}$ so in order to get the correct gain you have to calculate $R_out$ considering last stage output resistance. this can be known in cascode amplifier (cascaded triodes).
\begin{circuitikz}
\tikzstyle{every node}=[font=\small]


%-------------------------------------------------------------------------%
%---------------------------------Parameters------------------------------%
%-------------------------------------------------------------------------%


\def \BiasSpace {1}
\def \MosSpace {1.5}
\def \MosBase {0}
\def \AnSize {0.5}
\pgfmathparse{\AnSize/2}
\let \AnSpace \pgfmathresult
% \pgfmathparse{\AnSize - 0.1pt}
% \let \VccLabelSpace \pgfmathresult

\coordinate (NMOS1) at (0,\MosBase);
\coordinate (NMOS2) at ($ (NMOS1) + (0,\MosSpace) $);


\newcommand{\neTondComment}[2]{\draw [Stealth-] (#1) ++ (+\AnSpace, 0) -- ++(0,+\AnSize) -- ++(+\AnSize,0)  node [anchor=west] {#2}};
\newcommand{\nwTondComment}[2]{\draw [Stealth-] (#1) ++ (-\AnSpace, 0) -- ++(0,+\AnSize) -- ++(-\AnSize,0)  node [anchor=east] {#2}};
\newcommand{\seTondComment}[2]{\draw [Stealth-] (#1) ++ (+\AnSpace, 0) -- ++(0,-\AnSize) -- ++(+\AnSize,0)  node [anchor=west] {#2}};
\newcommand{\swTondComment}[2]{\draw [Stealth-] (#1) ++ (-\AnSpace, 0) -- ++(0,-\AnSize) -- ++(-\AnSize,0)  node [anchor=east] {#2}};

\newcommand{\neTondCommentH}[2]{\draw [Stealth-] (#1) ++ (0, +\AnSpace) -- ++(+\AnSize,0) -- ++(0,+\AnSize)  node [anchor=south] {#2}};
\newcommand{\nwTondCommentH}[2]{\draw [Stealth-] (#1) ++ (0, +\AnSpace) -- ++(-\AnSize,0) -- ++(0,+\AnSize)  node [anchor=south] {#2}};
\newcommand{\seTondCommentH}[2]{\draw [Stealth-] (#1) ++ (0, -\AnSpace) -- ++(+\AnSize,0) -- ++(0,-\AnSize)  node [anchor=north] {#2}};
\newcommand{\swTondCommentH}[2]{\draw [Stealth-] (#1) ++ (0, -\AnSpace) -- ++(-\AnSize,0) -- ++(0,-\AnSize)  node [anchor=north] {#2}};
%\newcommand{\myfrac}[3][0pt]{\genfrac{}{}{}{}{\raisebox{#1}{$#2$}}{\raisebox{-#1}{$#3$}}}

\def \TondVcc{node [anchor=south] {$V_{CC}$} ++(-\VccLabelSpace,0) --  ++(2*\VccLabelSpace,0)}%VCC
\newcommand{\TondOut}[1]{\node [shape=circle,anchor=west,draw,minimum size=0.1cm,inner sep=0pt] at (#1) {}};
\newcommand{\TondOutE}[1]{\node [shape=circle,anchor=east,draw,minimum size=0.1cm,inner sep=0pt] at (#1) {}};
\newcommand{\TondNode}[3]{
\node [shape=circle,minimum size=0.1cm,inner sep=0pt,fill=black] at (#1) {};
\ifthenelse{\equal{⟨#3⟩}{⟨n⟩}}{\node [anchor=south] at (#1) {#2};}{}
\ifthenelse{\equal{⟨#3⟩}{⟨e⟩}}{\node [anchor=west] at (#1) {#2};}{}
\ifthenelse{\equal{⟨#3⟩}{⟨s⟩}}{\node [anchor=north] at (#1) {#2};}{}
\ifthenelse{\equal{⟨#3⟩}{⟨w⟩}}{\node [anchor=east] at (#1) {#2};}{}
\equal{⟨string⟩}{⟨string⟩}
}

%Draw input signal generator on right of node
\newcommand{\TondSigR}[1]{
\coordinate (C_IN) at  ($ (#1) + (\SmallElement,0) $);
\coordinate (R_IN) at  ($ (C_IN) +  (\SmallElement,0) $);
\coordinate (GND_SIG) at  ($ (R_IN)  - (0,2) $);

\draw (GND_SIG) node [ground] {};%GND
\draw (C_IN) to[C=$C$] (#1);%Rsig
\draw (R_IN) to [R=$R_{sig}$] ($(C_IN) -  (0.4,0) $);%Rsig
\draw (GND_SIG) to [sV=$V_{in}$] (R_IN);%Signal
}
%Draw input signal generator on left of node
\newcommand{\TondSigL}[1]{
\coordinate (C_IN) at  ($ (#1) - (\SmallElement,0) $);
\coordinate (R_IN) at  ($ (C_IN) -  (\SmallElement,0) $);
\coordinate (GND_SIG) at  ($ (R_IN)  - (0,2) $);

\draw (GND_SIG) node [ground] {};%GND
\draw (C_IN) to[C=$C$] (#1);%Rsig
\draw (R_IN) to [R=$R_{sig}$] ($(C_IN) +  (0.4,0) $);%Rsig
\draw (GND_SIG) to [sV=$V_{in}$] (R_IN);%Signal
}



%-------------------------------------------------------------------------%
%-----------------------------------Circuit-------------------------------%
%-------------------------------------------------------------------------%


\draw (NMOS1) node[nmos](mos1) {};
\node [anchor=west] at (NMOS1) {M1};
\draw (NMOS2) node[nmos](mos2) {};
\TondNode{mos1.drain}{$V_X$}{e};
\node [anchor=west] at (NMOS2) {M2};
\node  [ground] at (mos1.source){};%GND

%----------------Bias----------------
\coordinate (Bias1) at ($ (mos1.gate) + (-\BiasSpace,0) $);
\coordinate (Bias2) at ($ (mos2.gate) + (-\BiasSpace,0) $);


\draw (mos1.gate)  -- (Bias1);
\node [anchor=east] at ($(Bias1) + (-0.3,0)$) {$V_{in}$};%Rl
\TondOutE {Bias1};

\draw (mos2.gate)  -- (Bias2);
\node [anchor=east] at ($(Bias2) + (-0.3,0)$) {$V_{G1}$};%Rl
\TondOutE {Bias2};


%-----------------Out----------------
\draw (mos2.drain) -- ++(0.5,0) coordinate (OUT1);%Rl
\node [anchor=west] at ($(OUT1) + (0.3,0)$) {$V_{O}$};%Rl
\TondOut {OUT1};

%-------------Annotate---------------
\nwTondComment{mos2.drain}{$R_{out}$};

\end{circuitikz}

\subsection{Current Mirrors}
\subsubsection{Wilson}
$$ 3~Mosfet ~~ , ~~ I_C = \frac{I_{Ref}}{1 + 2 \beta} $$
\par
\subsubsection{Widlar}
$$ 3~Mosfet ~~ , ~~ I_C = \frac{I_{Ref}}{1 + 2 \beta} $$
\par
\subsection{Feedback}
\subsubsection{Current Feedback}
Input Case: To inject or draw currents into or from inputs the feedback network has to be shunt with amplifier.\\
Output Case: To sample currents from output we want to read current ( like amp meter ) so it has to be series with output.\\
\lipsum[7-8]
\subsubsection{Voltage Feedback}
Input Case: To create voltage control amplifier the feedback network has to be series with signal to change input voltage.\\
Output Case: To sample voltage from output we want act like a voltage meter so it has to be shunt with output.
\par
\setlength{\parindent}{0.5cm} % Default is 15pt.
In Common Drain, $R_D$ Is not used as it has very small impact on output gain, but it take down DC bias. so it not used practically
\lipsum[1]
\setlength{\parindent}{0.0cm} 
