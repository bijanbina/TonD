\subsection{Multistage Amplifiers}
To calculate multistage gain for a amplifier. first calculate gain from V_{in} to V_x considering next stage load. then calculate $R_{out}$ of input stage by disconnecting second stage. then put $R_{out}$ series with a voltage source showing amplify voltage. For calcualting next stage gain note that in the nominator of CE, CB and their MOSFET configuration equivalent, the R_C is $_{out}$ so in order to get the correct gain you have to calculate $R_out$ considering last stage output resistance. this can be known in cascode amplifier (cascaded triodes).
\subsection{Current Mirrors}
\subsubsection{Wilson}
$$ 3~Mosfet ~~ , ~~ I_C = \frac{I_{Ref}}{1 + 2 \beta} $$
\par
\subsubsection{Widlar}
$$ 3~Mosfet ~~ , ~~ I_C = \frac{I_{Ref}}{1 + 2 \beta} $$
\par
\subsection{Feedback}
\subsubsection{Current Feedback}
Input Case: To inject or draw currents into or from inputs the feedback network has to be shunt with amplifier.\\
Output Case: To sample currents from output we want to read current ( like amp meter ) so it has to be series with output.\\
\lipsum[7-8]
\subsubsection{Voltage Feedback}
Input Case: To create voltage control amplifier the feedback network has to be series with signal to change input voltage.\\
Output Case: To sample voltage from output we want act like a voltage meter so it has to be shunt with output.
\par
\setlength{\parindent}{0.5cm} % Default is 15pt.
In Common Drain, $R_D$ Is not used as it has very small impact on output gain, but it take down DC bias. so it not used practically
\lipsum[1]
\setlength{\parindent}{0.0cm} 
