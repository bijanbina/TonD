%\thefontsize 
NMOS
$$ k = {\mu}_n C_{ox} \frac{W}{L} ~~ , ~~ V_{ov} = V_{GS} - V_{TH} ~~, ~~  Saturation: I_D = \frac{k}{2} {V_{ov}}^2 ~~,~~ Triode = k [(V_{GS} - V_{TH})V_{DS} - \frac{1}{2}{V_{DS}}^2]$$
PMOS
$$ k^{`}_p = {\mu}_p C_{ox} \frac{W}{L} ~~, ~~ I_D = \frac{k}{2} (V_{SG} - |V_{TH}|)^2 ~~,~~ 
\lambda = \frac{1}{V_A} ~~, ~~ r_o = \frac{\partial V_{DS}}{\partial I_{DS}} = \frac{1}{\lambda  I_D} = \frac{V_A}{I_D}$$
\par Common Source
$$ g_m = \frac{\partial I_D}{\partial V_{GS}} = k V_{ov} = \frac{2 I_D}{V_{ov}} =  \sqrt{2 k I_D}  $$
$$ A_V = -\dfrac{R_D || R_L}{g_m^{-1} + R_S} ~~ , ~~  {\Delta}_1 = V_{DSQ} - V_{ov} ~~ , ~~ I_{DQ} - \frac{{\Delta}_2}{R_{ac}} = 0 $$
\setlength{\parindent}{0.5cm} % Default is 15pt.
\par
Common x is an amplifing configuration which the node x, is isolated from both input source and output node. To compute swing voltage this node and the output node are plotted versus $I_C$. This is done to independ the swing calculation from input alternate. So as said in Common Source configuration this is $I_D$ versus $V_DS$. After writing DC load line from mentioned nodes. We determine AC Line, Slope. So we define AC Line. AC Line is a Line with "AC Line Slope" Slope and pass from Quiscent point. so for common emitter configuration it is $ I_C = - ( R_D || R_L )^{-1} V_{CE} $
\setlength{\parindent}{0.0cm} 
\begin{minipage}[b]{.5\textwidth}
$ R_o = (r_\pi ~||~ R_E) + ( 1 + g_m (r_\pi ~||~ R_E) )r_o$ or \\
$ R_o = r_o + (r_\pi ~||~ R_E)( 1 + g_m  )$ \\
$ if ( g_m R_o = r_o + (r_\pi ~||~ R_E)( 1 + g_m  )$ \\
\par 
\begin{circuitikz}
\tikzstyle{every node}=[font=\small]


%-------------------------------------------------------------------------%
%---------------------------------Parameters------------------------------%
%-------------------------------------------------------------------------%


\def \BiasSpace {1}
\def \LevelSpace {5}
\def \MosBase {0}
% \pgfmathparse{\VccSpace-1}
% \let \VccToSig \pgfmathresult
% 
% \ifx\MosOffset\undefined
% \def \MosOffset {0}
% \fi
% \pgfmathparse{\VccSpace/4}
% \let \SmallElement \pgfmathresult
% 
% \pgfmathparse{\VccSpace/2 + 0.5 + \MosOffset}
% \let \VccToMos \pgfmathresult %pgfmathparse used to make expression unify
% %\def \NormElement {\VccSpace/2 + 0.5}
\def \AnSize {0.5}
\pgfmathparse{\AnSize/2}
\let \AnSpace \pgfmathresult
% \pgfmathparse{\AnSize - 0.1pt}
% \let \VccLabelSpace \pgfmathresult

\coordinate (NMOS1) at (0,\LevelSpace);
\coordinate (PMOS1) at ($ (NMOS1) + (\LevelSpace,0) $);


\newcommand{\neTondComment}[2]{\draw [Stealth-] (#1) ++ (+\AnSpace, 0) -- ++(0,+\AnSize) -- ++(+\AnSize,0)  node [anchor=west] {#2}};
\newcommand{\nwTondComment}[2]{\draw [Stealth-] (#1) ++ (-\AnSpace, 0) -- ++(0,+\AnSize) -- ++(-\AnSize,0)  node [anchor=east] {#2}};
\newcommand{\seTondComment}[2]{\draw [Stealth-] (#1) ++ (+\AnSpace, 0) -- ++(0,-\AnSize) -- ++(+\AnSize,0)  node [anchor=west] {#2}};
\newcommand{\swTondComment}[2]{\draw [Stealth-] (#1) ++ (-\AnSpace, 0) -- ++(0,-\AnSize) -- ++(-\AnSize,0)  node [anchor=east] {#2}};

\newcommand{\neTondCommentH}[2]{\draw [Stealth-] (#1) ++ (0, +\AnSpace) -- ++(+\AnSize,0) -- ++(0,+\AnSize)  node [anchor=south] {#2}};
\newcommand{\nwTondCommentH}[2]{\draw [Stealth-] (#1) ++ (0, +\AnSpace) -- ++(-\AnSize,0) -- ++(0,+\AnSize)  node [anchor=south] {#2}};
\newcommand{\seTondCommentH}[2]{\draw [Stealth-] (#1) ++ (0, -\AnSpace) -- ++(+\AnSize,0) -- ++(0,-\AnSize)  node [anchor=north] {#2}};
\newcommand{\swTondCommentH}[2]{\draw [Stealth-] (#1) ++ (0, -\AnSpace) -- ++(-\AnSize,0) -- ++(0,-\AnSize)  node [anchor=north] {#2}};
%\newcommand{\myfrac}[3][0pt]{\genfrac{}{}{}{}{\raisebox{#1}{$#2$}}{\raisebox{-#1}{$#3$}}}

\def \TondVcc{node [anchor=south] {$V_{CC}$} ++(-\VccLabelSpace,0) --  ++(2*\VccLabelSpace,0)}%VCC
\newcommand{\TondOut}[1]{\node [shape=circle,anchor=west,draw,minimum size=0.1cm,inner sep=0pt] at (#1) {}};
\newcommand{\TondOutE}[1]{\node [shape=circle,anchor=east,draw,minimum size=0.1cm,inner sep=0pt] at (#1) {}};
\newcommand{\TondNode}[3]{
\node [shape=circle,minimum size=0.1cm,inner sep=0pt,fill=black] at (#1) {};
\ifthenelse{\equal{⟨#3⟩}{⟨n⟩}}{\node [anchor=south] at (#1) {#2};}{}
\ifthenelse{\equal{⟨#3⟩}{⟨e⟩}}{\node [anchor=west] at (#1) {#2};}{}
\ifthenelse{\equal{⟨#3⟩}{⟨s⟩}}{\node [anchor=north] at (#1) {#2};}{}
\ifthenelse{\equal{⟨#3⟩}{⟨w⟩}}{\node [anchor=east] at (#1) {#2};}{}
\equal{⟨string⟩}{⟨string⟩}
}

%Draw input signal generator on right of node
\newcommand{\TondSigR}[1]{
\coordinate (C_IN) at  ($ (#1) + (\SmallElement,0) $);
\coordinate (R_IN) at  ($ (C_IN) +  (\SmallElement,0) $);
\coordinate (GND_SIG) at  ($ (R_IN)  - (0,2) $);

\draw (GND_SIG) node [ground] {};%GND
\draw (C_IN) to[C=$C$] (#1);%Rsig
\draw (R_IN) to [R=$R_{sig}$] ($(C_IN) -  (0.4,0) $);%Rsig
\draw (GND_SIG) to [sV=$V_{in}$] (R_IN);%Signal
}
%Draw input signal generator on left of node
\newcommand{\TondSigL}[1]{
\coordinate (C_IN) at  ($ (#1) - (\SmallElement,0) $);
\coordinate (R_IN) at  ($ (C_IN) -  (\SmallElement,0) $);
\coordinate (GND_SIG) at  ($ (R_IN)  - (0,2) $);

\draw (GND_SIG) node [ground] {};%GND
\draw (C_IN) to[C=$C$] (#1);%Rsig
\draw (R_IN) to [R=$R_{sig}$] ($(C_IN) +  (0.4,0) $);%Rsig
\draw (GND_SIG) to [sV=$V_{in}$] (R_IN);%Signal
}



%-------------------------------------------------------------------------%
%-----------------------------------Circuit-------------------------------%
%-------------------------------------------------------------------------%


\draw (NMOS1) node[nmos](mos1) {};
\draw (mos1.gate) to [short,*-] ++ ( 0, 0.77 ) to [short,-*] (mos1.drain);
\draw (mos1.drain)  -- ++ ( 0, 0.25 ) coordinate (anot1);
\draw (PMOS1) node[pmos](mos2) {};
\draw (mos2.gate) to [short,*-] ++ ( 0, -0.77 ) to [short,-*] (mos2.drain);
\draw (mos2.drain)  -- ++ ( 0, -0.25 ) coordinate (anot2);


%----------------Bias----------------
\coordinate (Bias1) at ($ (mos1.gate) + (-\BiasSpace,0) $);
\coordinate (Bias2) at ($ (mos2.gate) + (-\BiasSpace,0) $);


\draw (mos1.gate) to node [anchor=north east] {$V_{bias}$} (Bias1);%Rl
\TondOutE {Bias1};

\draw (mos2.gate) to node [anchor=north east] {$V_{bias}$} (Bias2);%Rl
\TondOutE {Bias2};

%-------------Annotate---------------
\nwTondComment{anot1}{$\frac{1}{g_m}$};
\swTondComment{mos1.source}{$\frac{1}{g_m}$};
\nwTondCommentH{$ (mos1.gate) + (-0.5,0) $}{$\frac{1}{g_m}$};

\nwTondComment{mos2.source}{$\frac{1}{g_m}$};
\swTondComment{anot2}{$\frac{1}{g_m}$};
\nwTondCommentH{$ (mos2.gate) + (-0.5,0) $}{$\frac{1}{g_m}$};

\end{circuitikz}

\begin{tikzpicture}
\tikzset{>=latex}
%lines
\draw [firstLine, thick,name path=line 1] (0,4) node [anchor=west] {~~\footnotesize DC Load Line} -- (5,0);
\draw [secondLine, thick,name path=line 2] (0,3) -- (7,0) coordinate (D2) ++ (0,0.5) node [anchor=south] {\footnotesize AC Load Line};

%curve
\draw[thick, name path=VBE] (0,0.5) to [out=85,in=185] (1,3);% to [out=5,in=180] (7,3.2);
\draw[thick]  (1,3) -- +(1:6.5cm);

%Coordinate
\draw [<->, rounded corners, thick] (0,5) node [anchor=south] {$I_{D}$} -- (0,0) -- (8,0) node [anchor=north west] {$V_{DS}$};

\draw [name intersections={of=line 1 and line 2}] [help lines]  let \p1 = (intersection-1) in (intersection-1) -- ++(-\x1,0) coordinate (Icq) node [anchor=east] {$I_{DQ}$};
\draw [help lines]  let \p1 = (intersection-1) in (intersection-1) -- ++(0,-\y1) coordinate (Vceq) node [anchor=north] {$V_{DSQ}$};
\fill (intersection-1) circle (1.5pt);

\path [name path=A--B] (0.65,0) -| (0.65,3);
\path [name intersections={of=A--B and VBE}];
\draw [help lines]  (intersection-1) -- (0.65,0);
\node [anchor=north, help lines] at (0.65,0) {$V_{ov}$};


\Cote{(Vceq)}{(0.65,0)}{${\Delta}_1$}
\Cote{(Vceq)}{(D2)}{${\Delta}_2$}

\end{tikzpicture}\\
\end{minipage}%
\noindent
\setlength{\parindent}{0.0cm} 
\hspace*{-\parindent}%
\begin{minipage}[b]{.5\textwidth}
\centering
\begin{circuitikz}
\tikzstyle{every node}=[font=\small]


%-------------------------------------------------------------------------%
%---------------------------------Parameters------------------------------%
%-------------------------------------------------------------------------%


\def \BiasSpace {1}
\def \LevelSpace {5}
\def \MosBase {0}
% \pgfmathparse{\VccSpace-1}
% \let \VccToSig \pgfmathresult
% 
% \ifx\MosOffset\undefined
% \def \MosOffset {0}
% \fi
% \pgfmathparse{\VccSpace/4}
% \let \SmallElement \pgfmathresult
% 
% \pgfmathparse{\VccSpace/2 + 0.5 + \MosOffset}
% \let \VccToMos \pgfmathresult %pgfmathparse used to make expression unify
% %\def \NormElement {\VccSpace/2 + 0.5}
\def \AnSize {0.5}
\pgfmathparse{\AnSize/2}
\let \AnSpace \pgfmathresult
% \pgfmathparse{\AnSize - 0.1pt}
% \let \VccLabelSpace \pgfmathresult

\coordinate (NMOS1) at (0,\LevelSpace);
\coordinate (PMOS1) at ($ (NMOS1) + (\LevelSpace,0) $);


\newcommand{\neTondComment}[2]{\draw [Stealth-] (#1) ++ (+\AnSpace, 0) -- ++(0,+\AnSize) -- ++(+\AnSize,0)  node [anchor=west] {#2}};
\newcommand{\nwTondComment}[2]{\draw [Stealth-] (#1) ++ (-\AnSpace, 0) -- ++(0,+\AnSize) -- ++(-\AnSize,0)  node [anchor=east] {#2}};
\newcommand{\seTondComment}[2]{\draw [Stealth-] (#1) ++ (+\AnSpace, 0) -- ++(0,-\AnSize) -- ++(+\AnSize,0)  node [anchor=west] {#2}};
\newcommand{\swTondComment}[2]{\draw [Stealth-] (#1) ++ (-\AnSpace, 0) -- ++(0,-\AnSize) -- ++(-\AnSize,0)  node [anchor=east] {#2}};

\newcommand{\neTondCommentH}[2]{\draw [Stealth-] (#1) ++ (0, +\AnSpace) -- ++(+\AnSize,0) -- ++(0,+\AnSize)  node [anchor=south] {#2}};
\newcommand{\nwTondCommentH}[2]{\draw [Stealth-] (#1) ++ (0, +\AnSpace) -- ++(-\AnSize,0) -- ++(0,+\AnSize)  node [anchor=south] {#2}};
\newcommand{\seTondCommentH}[2]{\draw [Stealth-] (#1) ++ (0, -\AnSpace) -- ++(+\AnSize,0) -- ++(0,-\AnSize)  node [anchor=north] {#2}};
\newcommand{\swTondCommentH}[2]{\draw [Stealth-] (#1) ++ (0, -\AnSpace) -- ++(-\AnSize,0) -- ++(0,-\AnSize)  node [anchor=north] {#2}};
%\newcommand{\myfrac}[3][0pt]{\genfrac{}{}{}{}{\raisebox{#1}{$#2$}}{\raisebox{-#1}{$#3$}}}

\def \TondVcc{node [anchor=south] {$V_{CC}$} ++(-\VccLabelSpace,0) --  ++(2*\VccLabelSpace,0)}%VCC
\newcommand{\TondOut}[1]{\node [shape=circle,anchor=west,draw,minimum size=0.1cm,inner sep=0pt] at (#1) {}};
\newcommand{\TondOutE}[1]{\node [shape=circle,anchor=east,draw,minimum size=0.1cm,inner sep=0pt] at (#1) {}};
\newcommand{\TondNode}[3]{
\node [shape=circle,minimum size=0.1cm,inner sep=0pt,fill=black] at (#1) {};
\ifthenelse{\equal{⟨#3⟩}{⟨n⟩}}{\node [anchor=south] at (#1) {#2};}{}
\ifthenelse{\equal{⟨#3⟩}{⟨e⟩}}{\node [anchor=west] at (#1) {#2};}{}
\ifthenelse{\equal{⟨#3⟩}{⟨s⟩}}{\node [anchor=north] at (#1) {#2};}{}
\ifthenelse{\equal{⟨#3⟩}{⟨w⟩}}{\node [anchor=east] at (#1) {#2};}{}
\equal{⟨string⟩}{⟨string⟩}
}

%Draw input signal generator on right of node
\newcommand{\TondSigR}[1]{
\coordinate (C_IN) at  ($ (#1) + (\SmallElement,0) $);
\coordinate (R_IN) at  ($ (C_IN) +  (\SmallElement,0) $);
\coordinate (GND_SIG) at  ($ (R_IN)  - (0,2) $);

\draw (GND_SIG) node [ground] {};%GND
\draw (C_IN) to[C=$C$] (#1);%Rsig
\draw (R_IN) to [R=$R_{sig}$] ($(C_IN) -  (0.4,0) $);%Rsig
\draw (GND_SIG) to [sV=$V_{in}$] (R_IN);%Signal
}
%Draw input signal generator on left of node
\newcommand{\TondSigL}[1]{
\coordinate (C_IN) at  ($ (#1) - (\SmallElement,0) $);
\coordinate (R_IN) at  ($ (C_IN) -  (\SmallElement,0) $);
\coordinate (GND_SIG) at  ($ (R_IN)  - (0,2) $);

\draw (GND_SIG) node [ground] {};%GND
\draw (C_IN) to[C=$C$] (#1);%Rsig
\draw (R_IN) to [R=$R_{sig}$] ($(C_IN) +  (0.4,0) $);%Rsig
\draw (GND_SIG) to [sV=$V_{in}$] (R_IN);%Signal
}



%-------------------------------------------------------------------------%
%-----------------------------------Circuit-------------------------------%
%-------------------------------------------------------------------------%


\draw (NMOS1) node[nmos](mos1) {};
\draw (PMOS1) node[pmos](mos2) {};

%----------------Bias----------------
\coordinate (Bias1) at ($ (mos1.gate) + (-\BiasSpace,0) $);
\coordinate (Bias2) at ($ (mos2.gate) + (-\BiasSpace,0) $);


\draw (mos1.gate) to node [anchor=north east] {$V_{bias}$} (Bias1);%Rl
\TondOutE {Bias1};

\draw (mos2.gate) to node [anchor=north east] {$V_{bias}$} (Bias2);%Rl
\TondOutE {Bias2};

%-------------Annotate---------------
\nwTondComment{mos1.drain}{$r_o$};
\swTondComment{mos1.source}{$\frac{1}{g_m}$};
\nwTondCommentH{$ (mos1.gate) + (-0.5,0) $}{$\infty$};

\nwTondComment{mos2.source}{$\frac{1}{g_m}$};
\swTondComment{mos2.drain}{$r_o$};
\nwTondCommentH{$ (mos2.gate) + (-0.5,0) $}{$\infty$};

\end{circuitikz}

~\par ~\\ ~\\
{
\raggedleft
\begin{circuitikz}
\tikzstyle{every node}=[font=\small]


%-------------------------------------------------------------------------%
%---------------------------------Parameters------------------------------%
%-------------------------------------------------------------------------%


\def \VccSpace {6}
\def \LevelSpace {2.25}
\pgfmathparse{\VccSpace-1}
\let \VccToSig \pgfmathresult

\ifx\MosOffset\undefined
\def \MosOffset {0}
\fi
\pgfmathparse{\VccSpace/4}
\let \SmallElement \pgfmathresult

\pgfmathparse{\VccSpace/2 + 0.5 + \MosOffset}
\let \VccToMos \pgfmathresult %pgfmathparse used to make expression unify
%\def \NormElement {\VccSpace/2 + 0.5}
\def \AnSize {0.5}
\pgfmathparse{\AnSize/2}
\let \AnSpace \pgfmathresult
\pgfmathparse{\AnSize - 0.1pt}
\let \VccLabelSpace \pgfmathresult

\coordinate (VCC2) at (3,\VccSpace);
\coordinate (VCC1) at  ($ (VCC2) + (\LevelSpace,0) $);
\coordinate (GATE_IN) at  ($ (VCC2) - (0,\VccToMos) $);
\coordinate (GND1) at  ($ (VCC1) + (0,-\VccSpace) $);
\coordinate (GND2) at  ($ (VCC2) + (0,-\VccSpace) $);


\newcommand{\neTondComment}[2]{\draw [Stealth-] (#1) ++ (+\AnSpace, 0) -- ++(0,+\AnSize) -- ++(+\AnSize,0)  node [anchor=west] {#2}};
\newcommand{\nwTondComment}[2]{\draw [Stealth-] (#1) ++ (-\AnSpace, 0) -- ++(0,+\AnSize) -- ++(-\AnSize,0)  node [anchor=east] {#2}};
\newcommand{\seTondComment}[2]{\draw [Stealth-] (#1) ++ (+\AnSpace, 0) -- ++(0,-\AnSize) -- ++(+\AnSize,0)  node [anchor=west] {#2}};
\newcommand{\swTondComment}[2]{\draw [Stealth-] (#1) ++ (-\AnSpace, 0) -- ++(0,-\AnSize) -- ++(-\AnSize,0)  node [anchor=east] {#2}};

\newcommand{\neTondCommentH}[2]{\draw [Stealth-] (#1) ++ (0, +\AnSpace) -- ++(+\AnSize,0) -- ++(0,+\AnSize)  node [anchor=south] {#2}};
\newcommand{\nwTondCommentH}[2]{\draw [Stealth-] (#1) ++ (0, +\AnSpace) -- ++(-\AnSize,0) -- ++(0,+\AnSize)  node [anchor=south] {#2}};
\newcommand{\seTondCommentH}[2]{\draw [Stealth-] (#1) ++ (0, -\AnSpace) -- ++(+\AnSize,0) -- ++(0,-\AnSize)  node [anchor=north] {#2}};
\newcommand{\swTondCommentH}[2]{\draw [Stealth-] (#1) ++ (0, -\AnSpace) -- ++(-\AnSize,0) -- ++(0,-\AnSize)  node [anchor=north] {#2}};
%\newcommand{\myfrac}[3][0pt]{\genfrac{}{}{}{}{\raisebox{#1}{$#2$}}{\raisebox{-#1}{$#3$}}}

\def \TondVcc{node [anchor=south] {$V_{CC}$} ++(-\VccLabelSpace,0) --  ++(2*\VccLabelSpace,0)}%VCC
\newcommand{\TondOut}[1]{\node [shape=circle,anchor=west,draw,minimum size=0.1cm,inner sep=0pt] at (#1) {}};
\newcommand{\TondOutE}[1]{\node [shape=circle,anchor=east,draw,minimum size=0.1cm,inner sep=0pt] at (#1) {}};
\newcommand{\TondNode}[3]{
\node [shape=circle,minimum size=0.1cm,inner sep=0pt,fill=black] at (#1) {};
\ifthenelse{\equal{⟨#3⟩}{⟨n⟩}}{\node [anchor=south] at (#1) {#2};}{}
\ifthenelse{\equal{⟨#3⟩}{⟨e⟩}}{\node [anchor=west] at (#1) {#2};}{}
\ifthenelse{\equal{⟨#3⟩}{⟨s⟩}}{\node [anchor=north] at (#1) {#2};}{}
\ifthenelse{\equal{⟨#3⟩}{⟨w⟩}}{\node [anchor=east] at (#1) {#2};}{}
\equal{⟨string⟩}{⟨string⟩}
}

%Draw input signal generator on right of node
\newcommand{\TondSigR}[1]{
\coordinate (C_IN) at  ($ (#1) + (\SmallElement,0) $);
\coordinate (R_IN) at  ($ (C_IN) +  (\SmallElement,0) $);
\coordinate (GND_SIG) at  ($ (R_IN)  - (0,2) $);

\draw (GND_SIG) node [ground] {};%GND
\draw (C_IN) to[C=$C$] (#1);%Rsig
\draw (R_IN) to [R=$R_{sig}$] ($(C_IN) -  (0.4,0) $);%Rsig
\draw (GND_SIG) to [sV=$V_{in}$] (R_IN);%Signal
}
%Draw input signal generator on left of node
\newcommand{\TondSigL}[1]{
\coordinate (C_IN) at  ($ (#1) - (\SmallElement,0) $);
\coordinate (R_IN) at  ($ (C_IN) -  (\SmallElement,0) $);
\coordinate (GND_SIG) at  ($ (R_IN)  - (0,2) $);

\draw (GND_SIG) node [ground] {};%GND
\draw (C_IN) to[C=$C$] (#1);%Rsig
\draw (R_IN) to [R=$R_{sig}$] ($(C_IN) +  (0.4,0) $);%Rsig
\draw (GND_SIG) to [sV=$V_{in}$] (R_IN);%Signal
}



%-------------------------------------------------------------------------%
%-----------------------------------Circuit-------------------------------%
%-------------------------------------------------------------------------%


\draw (VCC2) \TondVcc;%VCC
\draw (VCC1) \TondVcc;%VCC
\draw (VCC1) ++ (0,-\VccToMos) node[nmos](mos) {};

\node  [ground] at (GND1){};%GND
\node  [ground] at (GND2){};%GND
\draw (mos.source) to[R=$R_S$] (GND1) {};%GND
\draw (VCC1) to [R=\small{$R_D$}] (mos.drain);%RD
\draw (VCC2) to [R=$R_1$] (GATE_IN) node(RG1) {};%RG1
\draw (RG1) to [R=$R_2$] (GND2);%RG2
\draw (mos.gate) -- (GATE_IN); %Short Circuit
%(mos.source) node[anchor=north] {S};
\nwTondComment{mos.drain}{$r_o$};
\swTondComment{mos.source}{$\frac{1}{g_m}$};

%----------------Signal--------------
\TondSigL{GATE_IN}

%-------------Load & Out-------------
\coordinate (V_out)  at  ($ (mos.drain)  + (\LevelSpace,0) $);
\coordinate (R_Load)  at  ($ (mos.drain)  + (\LevelSpace,0) - (1,0) $);
\coordinate (GND_Load) at  ($ (R_Load) - (0,2) $);

\draw (R_Load) to [R=$R_{l}$] (GND_Load);%Rl
\node  [ground] at (GND_Load){};%GND
\draw (mos.drain) to [C=$C$,*-] (R_Load) to[short] (V_out) node [anchor=north west] {$V_o$};%Rl
\TondOut {V_out};

\end{circuitikz}
}
%\makesectionhead{Mosfet}{Common source}
\end{minipage}%
